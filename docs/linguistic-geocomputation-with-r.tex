\documentclass[]{book}
\usepackage{lmodern}
\usepackage{amssymb,amsmath}
\usepackage{ifxetex,ifluatex}
\usepackage{fixltx2e} % provides \textsubscript
\ifnum 0\ifxetex 1\fi\ifluatex 1\fi=0 % if pdftex
  \usepackage[T1]{fontenc}
  \usepackage[utf8]{inputenc}
\else % if luatex or xelatex
  \ifxetex
    \usepackage{mathspec}
  \else
    \usepackage{fontspec}
  \fi
  \defaultfontfeatures{Ligatures=TeX,Scale=MatchLowercase}
\fi
% use upquote if available, for straight quotes in verbatim environments
\IfFileExists{upquote.sty}{\usepackage{upquote}}{}
% use microtype if available
\IfFileExists{microtype.sty}{%
\usepackage{microtype}
\UseMicrotypeSet[protrusion]{basicmath} % disable protrusion for tt fonts
}{}
\usepackage[margin=1in]{geometry}
\usepackage{hyperref}
\hypersetup{unicode=true,
            pdftitle={Linguistic Geocomputation with R},
            pdfauthor={George Moroz},
            pdfborder={0 0 0},
            breaklinks=true}
\urlstyle{same}  % don't use monospace font for urls
\usepackage{natbib}
\bibliographystyle{apalike}
\usepackage{color}
\usepackage{fancyvrb}
\newcommand{\VerbBar}{|}
\newcommand{\VERB}{\Verb[commandchars=\\\{\}]}
\DefineVerbatimEnvironment{Highlighting}{Verbatim}{commandchars=\\\{\}}
% Add ',fontsize=\small' for more characters per line
\usepackage{framed}
\definecolor{shadecolor}{RGB}{248,248,248}
\newenvironment{Shaded}{\begin{snugshade}}{\end{snugshade}}
\newcommand{\KeywordTok}[1]{\textcolor[rgb]{0.13,0.29,0.53}{\textbf{#1}}}
\newcommand{\DataTypeTok}[1]{\textcolor[rgb]{0.13,0.29,0.53}{#1}}
\newcommand{\DecValTok}[1]{\textcolor[rgb]{0.00,0.00,0.81}{#1}}
\newcommand{\BaseNTok}[1]{\textcolor[rgb]{0.00,0.00,0.81}{#1}}
\newcommand{\FloatTok}[1]{\textcolor[rgb]{0.00,0.00,0.81}{#1}}
\newcommand{\ConstantTok}[1]{\textcolor[rgb]{0.00,0.00,0.00}{#1}}
\newcommand{\CharTok}[1]{\textcolor[rgb]{0.31,0.60,0.02}{#1}}
\newcommand{\SpecialCharTok}[1]{\textcolor[rgb]{0.00,0.00,0.00}{#1}}
\newcommand{\StringTok}[1]{\textcolor[rgb]{0.31,0.60,0.02}{#1}}
\newcommand{\VerbatimStringTok}[1]{\textcolor[rgb]{0.31,0.60,0.02}{#1}}
\newcommand{\SpecialStringTok}[1]{\textcolor[rgb]{0.31,0.60,0.02}{#1}}
\newcommand{\ImportTok}[1]{#1}
\newcommand{\CommentTok}[1]{\textcolor[rgb]{0.56,0.35,0.01}{\textit{#1}}}
\newcommand{\DocumentationTok}[1]{\textcolor[rgb]{0.56,0.35,0.01}{\textbf{\textit{#1}}}}
\newcommand{\AnnotationTok}[1]{\textcolor[rgb]{0.56,0.35,0.01}{\textbf{\textit{#1}}}}
\newcommand{\CommentVarTok}[1]{\textcolor[rgb]{0.56,0.35,0.01}{\textbf{\textit{#1}}}}
\newcommand{\OtherTok}[1]{\textcolor[rgb]{0.56,0.35,0.01}{#1}}
\newcommand{\FunctionTok}[1]{\textcolor[rgb]{0.00,0.00,0.00}{#1}}
\newcommand{\VariableTok}[1]{\textcolor[rgb]{0.00,0.00,0.00}{#1}}
\newcommand{\ControlFlowTok}[1]{\textcolor[rgb]{0.13,0.29,0.53}{\textbf{#1}}}
\newcommand{\OperatorTok}[1]{\textcolor[rgb]{0.81,0.36,0.00}{\textbf{#1}}}
\newcommand{\BuiltInTok}[1]{#1}
\newcommand{\ExtensionTok}[1]{#1}
\newcommand{\PreprocessorTok}[1]{\textcolor[rgb]{0.56,0.35,0.01}{\textit{#1}}}
\newcommand{\AttributeTok}[1]{\textcolor[rgb]{0.77,0.63,0.00}{#1}}
\newcommand{\RegionMarkerTok}[1]{#1}
\newcommand{\InformationTok}[1]{\textcolor[rgb]{0.56,0.35,0.01}{\textbf{\textit{#1}}}}
\newcommand{\WarningTok}[1]{\textcolor[rgb]{0.56,0.35,0.01}{\textbf{\textit{#1}}}}
\newcommand{\AlertTok}[1]{\textcolor[rgb]{0.94,0.16,0.16}{#1}}
\newcommand{\ErrorTok}[1]{\textcolor[rgb]{0.64,0.00,0.00}{\textbf{#1}}}
\newcommand{\NormalTok}[1]{#1}
\usepackage{longtable,booktabs}
\usepackage{graphicx,grffile}
\makeatletter
\def\maxwidth{\ifdim\Gin@nat@width>\linewidth\linewidth\else\Gin@nat@width\fi}
\def\maxheight{\ifdim\Gin@nat@height>\textheight\textheight\else\Gin@nat@height\fi}
\makeatother
% Scale images if necessary, so that they will not overflow the page
% margins by default, and it is still possible to overwrite the defaults
% using explicit options in \includegraphics[width, height, ...]{}
\setkeys{Gin}{width=\maxwidth,height=\maxheight,keepaspectratio}
\IfFileExists{parskip.sty}{%
\usepackage{parskip}
}{% else
\setlength{\parindent}{0pt}
\setlength{\parskip}{6pt plus 2pt minus 1pt}
}
\setlength{\emergencystretch}{3em}  % prevent overfull lines
\providecommand{\tightlist}{%
  \setlength{\itemsep}{0pt}\setlength{\parskip}{0pt}}
\setcounter{secnumdepth}{5}
% Redefines (sub)paragraphs to behave more like sections
\ifx\paragraph\undefined\else
\let\oldparagraph\paragraph
\renewcommand{\paragraph}[1]{\oldparagraph{#1}\mbox{}}
\fi
\ifx\subparagraph\undefined\else
\let\oldsubparagraph\subparagraph
\renewcommand{\subparagraph}[1]{\oldsubparagraph{#1}\mbox{}}
\fi

%%% Use protect on footnotes to avoid problems with footnotes in titles
\let\rmarkdownfootnote\footnote%
\def\footnote{\protect\rmarkdownfootnote}

%%% Change title format to be more compact
\usepackage{titling}

% Create subtitle command for use in maketitle
\newcommand{\subtitle}[1]{
  \posttitle{
    \begin{center}\large#1\end{center}
    }
}

\setlength{\droptitle}{-2em}

  \title{Linguistic Geocomputation with R}
    \pretitle{\vspace{\droptitle}\centering\huge}
  \posttitle{\par}
    \author{George Moroz}
    \preauthor{\centering\large\emph}
  \postauthor{\par}
      \predate{\centering\large\emph}
  \postdate{\par}
    \date{2018-06-26}

\usepackage{booktabs}
\usepackage{amsthm}
\makeatletter
\def\thm@space@setup{%
  \thm@preskip=8pt plus 2pt minus 4pt
  \thm@postskip=\thm@preskip
}
\makeatother

\usepackage{amsthm}
\newtheorem{theorem}{Theorem}[chapter]
\newtheorem{lemma}{Lemma}[chapter]
\theoremstyle{definition}
\newtheorem{definition}{Definition}[chapter]
\newtheorem{corollary}{Corollary}[chapter]
\newtheorem{proposition}{Proposition}[chapter]
\theoremstyle{definition}
\newtheorem{example}{Example}[chapter]
\theoremstyle{definition}
\newtheorem{exercise}{Exercise}[chapter]
\theoremstyle{remark}
\newtheorem*{remark}{Remark}
\newtheorem*{solution}{Solution}
\begin{document}
\maketitle

{
\setcounter{tocdepth}{1}
\tableofcontents
}
This book is about

\chapter{Introduction}\label{intro}

\section{Why linguistic
geocomputations?}\label{why-linguistic-geocomputations}

\section{Why do we need geostatistics in
linguistics?}\label{why-do-we-need-geostatistics-in-linguistics}

\section{Why R?}\label{why-r}

\chapter{Introduction to R language}\label{introduction-to-r-language}

Since this book includes a lot of R code examples, this chapter will
describe some basics for those, who is not familiar with R. For purposes
of understanding R code in this book you don't need any deep knowledge
of R. In case you want to learn more, there are a lot of good books on
it. I will list only few of them:

\begin{itemize}
\item
\item
\end{itemize}

\section{Instalation}\label{instalation}

\subsection{R instalation}\label{r-instalation}

To download R, go to \href{https://cran.r-project.org/}{CRAN}. Don't try
to pick a mirror that's close to you, instead it is better to use the
cloud mirror, \url{https://cloud.r-project.org}.

\subsection{RStudio}\label{rstudio}

\href{https://www.rstudio.com/products/rstudio/download/}{RStudio} is an
integrated development environment, or IDE, for R programming. There are
two possibilities for using it:

\begin{itemize}
\tightlist
\item
  type R code in the R console pane, and press enter to run it;
\item
  type R code in the Code editor pane, and press Control/Command + Enter
  to run selected part. It is easier to correct and it is possible to
  save the result as a script.
\end{itemize}

\begin{figure}

{\centering \includegraphics[width=5in]{images/02-rstudio} 

}

\caption{RStudio layout}\label{fig:rstudio}
\end{figure}

When you first launch RStudio it is more likely, that you won't see the
Code Editor pane. It is possible to decrease R Console pane on icons in
the pane's right upper corner.

Everything from this book will be availible without RStudio instalation.
There are a lot of possibilities to work with R not using RStudio such
as R console, command line, Jupyter Notebook, some plugins for working
in Sublime, Vim, Emacs, Atom, Notepad++ and other programming text
editors.

\subsection{RStuio cloud}\label{rstuio-cloud}

It is also possible not to install anything on your own PC, using
\href{https://rstudio.cloud/}{RStudio Cloud}, a web-based interface for
Rstudio and R. In RStudio Cloud it is also possible to share your R
projects and collaborate with a select group in a private space. RStudio
Cloud is currently free to use, but soon there will be free and paid
options.

\section{Basic elements, variables, vectors,
dataframe}\label{basic-elements-variables-vectors-dataframe}

\subsection{Basic elements}\label{basic-elements}

\begin{Shaded}
\begin{Highlighting}[]
\DecValTok{7}
\end{Highlighting}
\end{Shaded}

\begin{verbatim}
[1] 7
\end{verbatim}

\begin{Shaded}
\begin{Highlighting}[]
\OperatorTok{-}\FloatTok{5.7}
\end{Highlighting}
\end{Shaded}

\begin{verbatim}
[1] -5.7
\end{verbatim}

\begin{Shaded}
\begin{Highlighting}[]
\StringTok{"bonjour"}
\end{Highlighting}
\end{Shaded}

\begin{verbatim}
[1] "bonjour"
\end{verbatim}

\begin{Shaded}
\begin{Highlighting}[]
\StringTok{"bon mot"}
\end{Highlighting}
\end{Shaded}

\begin{verbatim}
[1] "bon mot"
\end{verbatim}

\begin{Shaded}
\begin{Highlighting}[]
\OtherTok{TRUE}
\end{Highlighting}
\end{Shaded}

\begin{verbatim}
[1] TRUE
\end{verbatim}

\begin{Shaded}
\begin{Highlighting}[]
\OtherTok{FALSE}
\end{Highlighting}
\end{Shaded}

\begin{verbatim}
[1] FALSE
\end{verbatim}

\subsection{Arithmetic operations}\label{arithmetic-operations}

\begin{Shaded}
\begin{Highlighting}[]
\DecValTok{7}\OperatorTok{+}\DecValTok{7}
\end{Highlighting}
\end{Shaded}

\begin{verbatim}
[1] 14
\end{verbatim}

\begin{Shaded}
\begin{Highlighting}[]
\DecValTok{21}\OperatorTok{-}\DecValTok{8}
\end{Highlighting}
\end{Shaded}

\begin{verbatim}
[1] 13
\end{verbatim}

\begin{Shaded}
\begin{Highlighting}[]
\DecValTok{4}\OperatorTok{*}\DecValTok{3}
\end{Highlighting}
\end{Shaded}

\begin{verbatim}
[1] 12
\end{verbatim}

\begin{Shaded}
\begin{Highlighting}[]
\DecValTok{12}\OperatorTok{/}\DecValTok{4}
\end{Highlighting}
\end{Shaded}

\begin{verbatim}
[1] 3
\end{verbatim}

\begin{Shaded}
\begin{Highlighting}[]
\DecValTok{4}\OperatorTok{^}\DecValTok{3}
\end{Highlighting}
\end{Shaded}

\begin{verbatim}
[1] 64
\end{verbatim}

\begin{Shaded}
\begin{Highlighting}[]
\DecValTok{4}\OperatorTok{**}\DecValTok{3}
\end{Highlighting}
\end{Shaded}

\begin{verbatim}
[1] 64
\end{verbatim}

\begin{Shaded}
\begin{Highlighting}[]
\KeywordTok{sum}\NormalTok{(}\DecValTok{1}\NormalTok{, }\DecValTok{2}\NormalTok{,}\DecValTok{3}\NormalTok{, }\DecValTok{4}\NormalTok{)}
\end{Highlighting}
\end{Shaded}

\begin{verbatim}
[1] 10
\end{verbatim}

\begin{Shaded}
\begin{Highlighting}[]
\KeywordTok{prod}\NormalTok{(}\DecValTok{1}\NormalTok{, }\DecValTok{2}\NormalTok{,}\DecValTok{3}\NormalTok{, }\DecValTok{4}\NormalTok{)}
\end{Highlighting}
\end{Shaded}

\begin{verbatim}
[1] 24
\end{verbatim}

\begin{Shaded}
\begin{Highlighting}[]
\KeywordTok{log}\NormalTok{(}\DecValTok{1}\NormalTok{)}
\end{Highlighting}
\end{Shaded}

\begin{verbatim}
[1] 0
\end{verbatim}

\begin{Shaded}
\begin{Highlighting}[]
\KeywordTok{log}\NormalTok{(}\DecValTok{100}\NormalTok{, }\DataTypeTok{base =} \DecValTok{10}\NormalTok{)}
\end{Highlighting}
\end{Shaded}

\begin{verbatim}
[1] 2
\end{verbatim}

\begin{Shaded}
\begin{Highlighting}[]
\NormalTok{pi}
\end{Highlighting}
\end{Shaded}

\begin{verbatim}
[1] 3.141593
\end{verbatim}

\begin{Shaded}
\begin{Highlighting}[]
\KeywordTok{exp}\NormalTok{(}\DecValTok{5}\NormalTok{)}
\end{Highlighting}
\end{Shaded}

\begin{verbatim}
[1] 148.4132
\end{verbatim}

\begin{Shaded}
\begin{Highlighting}[]
\KeywordTok{sin}\NormalTok{(}\DecValTok{13}\NormalTok{)}
\end{Highlighting}
\end{Shaded}

\begin{verbatim}
[1] 0.420167
\end{verbatim}

\begin{Shaded}
\begin{Highlighting}[]
\KeywordTok{cos}\NormalTok{(}\DecValTok{13}\NormalTok{)}
\end{Highlighting}
\end{Shaded}

\begin{verbatim}
[1] 0.9074468
\end{verbatim}

\subsection{Variables}\label{variables}

\begin{Shaded}
\begin{Highlighting}[]
\NormalTok{my_var <-}\StringTok{ }\DecValTok{7}
\NormalTok{my_var}
\end{Highlighting}
\end{Shaded}

\begin{verbatim}
[1] 7
\end{verbatim}

\begin{Shaded}
\begin{Highlighting}[]
\NormalTok{my_var}\OperatorTok{+}\DecValTok{7}
\end{Highlighting}
\end{Shaded}

\begin{verbatim}
[1] 14
\end{verbatim}

\begin{Shaded}
\begin{Highlighting}[]
\NormalTok{my_var}
\end{Highlighting}
\end{Shaded}

\begin{verbatim}
[1] 7
\end{verbatim}

\begin{Shaded}
\begin{Highlighting}[]
\NormalTok{my_var <-}\StringTok{ }\NormalTok{my_var }\OperatorTok{+}\StringTok{ }\DecValTok{7}
\end{Highlighting}
\end{Shaded}

\subsection{Vectors}\label{vectors}

\begin{Shaded}
\begin{Highlighting}[]
\DecValTok{5}\OperatorTok{:}\DecValTok{9}
\end{Highlighting}
\end{Shaded}

\begin{verbatim}
[1] 5 6 7 8 9
\end{verbatim}

\begin{Shaded}
\begin{Highlighting}[]
\DecValTok{11}\OperatorTok{:}\DecValTok{4}
\end{Highlighting}
\end{Shaded}

\begin{verbatim}
[1] 11 10  9  8  7  6  5  4
\end{verbatim}

\begin{Shaded}
\begin{Highlighting}[]
\NormalTok{numbers <-}\StringTok{ }\KeywordTok{c}\NormalTok{(}\DecValTok{7}\NormalTok{, }\FloatTok{9.9}\NormalTok{, }\DecValTok{24}\NormalTok{)}
\NormalTok{multiple_strings <-}\StringTok{ }\KeywordTok{c}\NormalTok{(}\StringTok{"the"}\NormalTok{, }\StringTok{"quick"}\NormalTok{, }\StringTok{"brown"}\NormalTok{, }\StringTok{"fox"}\NormalTok{, }\StringTok{"jumps"}\NormalTok{, }\StringTok{"over"}\NormalTok{, }\StringTok{"the"}\NormalTok{, }\StringTok{"lazy"}\NormalTok{, }\StringTok{"dog"}\NormalTok{)}
\NormalTok{one_string <-}\StringTok{ }\KeywordTok{c}\NormalTok{(}\StringTok{"the quick brown fox jumps over the lazy dog"}\NormalTok{)}
\NormalTok{true_false <-}\StringTok{ }\KeywordTok{c}\NormalTok{(}\OtherTok{TRUE}\NormalTok{, }\OtherTok{FALSE}\NormalTok{, }\OtherTok{FALSE}\NormalTok{, }\OtherTok{TRUE}\NormalTok{)}
\KeywordTok{length}\NormalTok{(numbers)}
\end{Highlighting}
\end{Shaded}

\begin{verbatim}
[1] 3
\end{verbatim}

\begin{Shaded}
\begin{Highlighting}[]
\KeywordTok{length}\NormalTok{(multiple_strings)}
\end{Highlighting}
\end{Shaded}

\begin{verbatim}
[1] 9
\end{verbatim}

\begin{Shaded}
\begin{Highlighting}[]
\KeywordTok{length}\NormalTok{(one_string)}
\end{Highlighting}
\end{Shaded}

\begin{verbatim}
[1] 1
\end{verbatim}

\subsection{Dataframes}\label{dataframes}

\begin{Shaded}
\begin{Highlighting}[]
\NormalTok{my_df <-}\StringTok{ }\KeywordTok{data.frame}\NormalTok{(}\DataTypeTok{latin =} \KeywordTok{c}\NormalTok{(}\StringTok{"a"}\NormalTok{, }\StringTok{"b"}\NormalTok{, }\StringTok{"c"}\NormalTok{),}
                    \DataTypeTok{cyrillic =} \KeywordTok{c}\NormalTok{(}\StringTok{"а"}\NormalTok{, }\StringTok{"б"}\NormalTok{, }\StringTok{"в"}\NormalTok{),}
                    \DataTypeTok{greek =} \KeywordTok{c}\NormalTok{(}\StringTok{"α"}\NormalTok{, }\StringTok{"β"}\NormalTok{, }\StringTok{"γ"}\NormalTok{),}
                    \DataTypeTok{numbers =} \KeywordTok{c}\NormalTok{(}\DecValTok{1}\OperatorTok{:}\DecValTok{3}\NormalTok{),}
                    \DataTypeTok{is.vowel =} \KeywordTok{c}\NormalTok{(}\OtherTok{TRUE}\NormalTok{, }\OtherTok{FALSE}\NormalTok{, }\OtherTok{FALSE}\NormalTok{),}
                    \DataTypeTok{stringsAsFactors =} \OtherTok{FALSE}\NormalTok{)}
\NormalTok{my_df}
\end{Highlighting}
\end{Shaded}

\begin{verbatim}
  latin cyrillic greek numbers is.vowel
1     a        а     α       1     TRUE
2     b        б     β       2    FALSE
3     c        в     γ       3    FALSE
\end{verbatim}

\begin{Shaded}
\begin{Highlighting}[]
\KeywordTok{nrow}\NormalTok{(my_df)}
\end{Highlighting}
\end{Shaded}

\begin{verbatim}
[1] 3
\end{verbatim}

\begin{Shaded}
\begin{Highlighting}[]
\KeywordTok{ncol}\NormalTok{(my_df)}
\end{Highlighting}
\end{Shaded}

\begin{verbatim}
[1] 5
\end{verbatim}

\subsection{Indexing}\label{indexing}

\begin{Shaded}
\begin{Highlighting}[]
\NormalTok{numbers[}\DecValTok{3}\NormalTok{]}
\end{Highlighting}
\end{Shaded}

\begin{verbatim}
[1] 24
\end{verbatim}

\begin{Shaded}
\begin{Highlighting}[]
\NormalTok{multiple_strings[}\DecValTok{9}\NormalTok{]}
\end{Highlighting}
\end{Shaded}

\begin{verbatim}
[1] "dog"
\end{verbatim}

\begin{Shaded}
\begin{Highlighting}[]
\NormalTok{my_df[}\DecValTok{2}\NormalTok{, }\DecValTok{3}\NormalTok{]}
\end{Highlighting}
\end{Shaded}

\begin{verbatim}
[1] "β"
\end{verbatim}

\begin{Shaded}
\begin{Highlighting}[]
\NormalTok{my_df[}\DecValTok{2}\NormalTok{,]}
\end{Highlighting}
\end{Shaded}

\begin{verbatim}
  latin cyrillic greek numbers is.vowel
2     b        б     β       2    FALSE
\end{verbatim}

\begin{Shaded}
\begin{Highlighting}[]
\NormalTok{my_df[,}\DecValTok{3}\NormalTok{]}
\end{Highlighting}
\end{Shaded}

\begin{verbatim}
[1] "α" "β" "γ"
\end{verbatim}

\begin{Shaded}
\begin{Highlighting}[]
\NormalTok{my_df}\OperatorTok{$}\NormalTok{is.vowel}
\end{Highlighting}
\end{Shaded}

\begin{verbatim}
[1]  TRUE FALSE FALSE
\end{verbatim}

\begin{Shaded}
\begin{Highlighting}[]
\NormalTok{my_df}\OperatorTok{$}\NormalTok{is.vowel[}\DecValTok{2}\NormalTok{]}
\end{Highlighting}
\end{Shaded}

\begin{verbatim}
[1] FALSE
\end{verbatim}

\section{Reading files}\label{reading-files}

We can read to R a dataset about Numeral Classifiers from
\href{https://github.com/autotyp/autotyp-data}{AUTOTYP database}.

\begin{Shaded}
\begin{Highlighting}[]
\NormalTok{new_df <-}\StringTok{ }\KeywordTok{read.csv}\NormalTok{(}\StringTok{"https://raw.githubusercontent.com/autotyp/autotyp-data/master/data/Numeral_classifiers.csv"}\NormalTok{)}
\KeywordTok{head}\NormalTok{(new_df)}
\end{Highlighting}
\end{Shaded}

\begin{verbatim}
  LID NumClass.n NumClass.Presence
1 148          0             FALSE
2  65          0             FALSE
3  75          0             FALSE
4  85          0             FALSE
5 111         NA                NA
6 163          0             FALSE
\end{verbatim}

\begin{Shaded}
\begin{Highlighting}[]
\KeywordTok{tail}\NormalTok{(new_df)}
\end{Highlighting}
\end{Shaded}

\begin{verbatim}
     LID NumClass.n NumClass.Presence
250 1397          0             FALSE
251 2994          5              TRUE
252 2779          0             FALSE
253  192          0             FALSE
254  551          0             FALSE
255 2564          2              TRUE
\end{verbatim}

It could be also a file on your computer, just provide a whole path to
the file. Windows users need to change backslashes
\texttt{\textbackslash{}} to slashes \texttt{/}.

\begin{Shaded}
\begin{Highlighting}[]
\NormalTok{new_df_}\DecValTok{2}\NormalTok{ <-}\StringTok{ }\KeywordTok{read.csv}\NormalTok{(}\StringTok{"/home/agricolamz/my_file.csv"}\NormalTok{)}
\end{Highlighting}
\end{Shaded}

\section{Writing files from R}\label{writing-files-from-r}

\begin{Shaded}
\begin{Highlighting}[]
\KeywordTok{write.csv}\NormalTok{(new_df_}\DecValTok{2}\NormalTok{, }\StringTok{"/home/agricolamz/my_new_file.csv"}\NormalTok{,}
          \DataTypeTok{row.names =} \OtherTok{FALSE}\NormalTok{)}
\end{Highlighting}
\end{Shaded}

\section{Missing data}\label{missing-data}

In R, missing values are represented by the symbol \texttt{NA} (not
available).

\begin{Shaded}
\begin{Highlighting}[]
\KeywordTok{is.na}\NormalTok{(new_df}\OperatorTok{$}\NormalTok{NumClass.Presence)}
\end{Highlighting}
\end{Shaded}

\begin{verbatim}
  [1] FALSE FALSE FALSE FALSE  TRUE FALSE FALSE FALSE FALSE FALSE FALSE
 [12] FALSE FALSE FALSE FALSE FALSE FALSE FALSE FALSE FALSE FALSE FALSE
 [23] FALSE FALSE FALSE FALSE FALSE FALSE FALSE FALSE FALSE FALSE FALSE
 [34] FALSE FALSE FALSE FALSE FALSE FALSE FALSE FALSE FALSE FALSE FALSE
 [45] FALSE FALSE FALSE FALSE FALSE FALSE FALSE FALSE FALSE FALSE FALSE
 [56] FALSE FALSE FALSE FALSE FALSE FALSE FALSE FALSE FALSE FALSE FALSE
 [67] FALSE FALSE FALSE FALSE FALSE FALSE FALSE FALSE FALSE FALSE FALSE
 [78] FALSE FALSE FALSE FALSE FALSE FALSE FALSE FALSE FALSE FALSE FALSE
 [89] FALSE FALSE FALSE FALSE FALSE FALSE FALSE FALSE FALSE FALSE FALSE
[100] FALSE FALSE FALSE FALSE FALSE FALSE  TRUE FALSE FALSE FALSE FALSE
[111] FALSE FALSE FALSE FALSE FALSE FALSE FALSE FALSE FALSE FALSE FALSE
[122] FALSE FALSE FALSE FALSE FALSE FALSE FALSE FALSE FALSE FALSE FALSE
[133] FALSE FALSE FALSE FALSE FALSE FALSE FALSE FALSE FALSE FALSE FALSE
[144] FALSE FALSE FALSE FALSE FALSE FALSE FALSE FALSE FALSE FALSE FALSE
[155] FALSE FALSE FALSE FALSE FALSE FALSE FALSE FALSE FALSE FALSE  TRUE
[166] FALSE FALSE FALSE FALSE FALSE FALSE FALSE FALSE FALSE FALSE FALSE
[177]  TRUE FALSE FALSE FALSE FALSE FALSE FALSE FALSE  TRUE FALSE FALSE
[188] FALSE FALSE FALSE FALSE FALSE FALSE FALSE FALSE FALSE FALSE FALSE
[199] FALSE FALSE FALSE FALSE FALSE FALSE FALSE FALSE FALSE FALSE FALSE
[210] FALSE FALSE FALSE FALSE FALSE FALSE FALSE FALSE FALSE FALSE FALSE
[221] FALSE FALSE FALSE FALSE FALSE FALSE FALSE FALSE FALSE FALSE FALSE
[232] FALSE FALSE FALSE FALSE FALSE FALSE FALSE FALSE FALSE FALSE FALSE
[243] FALSE FALSE FALSE FALSE FALSE FALSE FALSE FALSE FALSE FALSE FALSE
[254] FALSE FALSE
\end{verbatim}

\begin{Shaded}
\begin{Highlighting}[]
\KeywordTok{sum}\NormalTok{(}\KeywordTok{is.na}\NormalTok{(new_df}\OperatorTok{$}\NormalTok{NumClass.Presence))}
\end{Highlighting}
\end{Shaded}

\begin{verbatim}
[1] 5
\end{verbatim}

\begin{Shaded}
\begin{Highlighting}[]
\KeywordTok{sum}\NormalTok{(}\KeywordTok{is.na}\NormalTok{(new_df))}
\end{Highlighting}
\end{Shaded}

\begin{verbatim}
[1] 22
\end{verbatim}

\section{How to get help in R}\label{how-to-get-help-in-r}

\begin{Shaded}
\begin{Highlighting}[]
\NormalTok{?nchar}
\end{Highlighting}
\end{Shaded}

\section{Packages}\label{packages}

There are a lot of R packages for solving a lot of different problems.
There are two way for install them (you need an internet connection):

\begin{itemize}
\tightlist
\item
  packages on CRAN are checked in multiple ways and should be stable
\end{itemize}

\begin{Shaded}
\begin{Highlighting}[]
\KeywordTok{install.packages}\NormalTok{(}\StringTok{"lingtypology"}\NormalTok{)}
\end{Highlighting}
\end{Shaded}

\begin{itemize}
\tightlist
\item
  packages on GitHub are NOT checked and could contain anything, but it
  is the place where all package developers keep the last vertion of
  they work.
\end{itemize}

\begin{Shaded}
\begin{Highlighting}[]
\KeywordTok{install.packages}\NormalTok{(}\StringTok{"devtools"}\NormalTok{)}
\NormalTok{devtools}\OperatorTok{::}\KeywordTok{install_github}\NormalTok{(}\StringTok{"ropensci/lingtypology"}\NormalTok{)}
\end{Highlighting}
\end{Shaded}

\begin{itemize}
\tightlist
\item
  or package file
\end{itemize}

\begin{Shaded}
\begin{Highlighting}[]
\KeywordTok{install.packages}\NormalTok{(}\StringTok{"lingtypology"}\NormalTok{,}
                 \DataTypeTok{destdir =} \StringTok{"/path/to/your/package"}\NormalTok{)}
\end{Highlighting}
\end{Shaded}

After the package is installed you need to load the package using the
following command:

\begin{Shaded}
\begin{Highlighting}[]
\KeywordTok{library}\NormalTok{(}\StringTok{"lingtypology"}\NormalTok{)}
\end{Highlighting}
\end{Shaded}

There is a nice picture from
\href{https://bookdown.org/ndphillips/YaRrr/}{Phillips N. D. (2017)
YaRrr! The Pirate's Guide to R}:

\begin{figure}

{\centering \includegraphics[width=6.89in]{images/02-package} 

}

\caption{Lamp metaphore}\label{fig:lamp}
\end{figure}

\chapter{Map creation}\label{map-creation}

\chapter{Linguistical databases}\label{db}

\section{Linguistical databases APIs}\label{api}

\section{Linguistical databases creation}\label{db-creation}

Look \ref{api} and \ref{map-creation}

\chapter{Spatial statistics}\label{statistics}

\section{Not all similarities are the
same}\label{not-all-similarities-are-the-same}

In previous chapters we have learnt how to visualize different
linguistic features on the map. But, as title says, not all similarities
are the same. Some languages share some features, because the trait was
inherited from a common ancestor. This type of similarities are called
\textbf{homologous}. For example the majority of the Slavic languages
(except Bulgarian and Macedonian) have a case system, because they
inherited this from Proto-Slavic. Despite coming from different
ancecsters some languages can independently evolve some analogous
traits. This type of similarities are called \textbf{analogous}. For
example despite Proto-Slavic, Latin and Proto-Germanic had case systems,
few languages such as Bulgarian, English and French independently lost
the case systems. Analogous similarities are commonly devided into
typological and areal.

\begin{figure}

{\centering \includegraphics{linguistic-geocomputation-with-r_files/figure-latex/unnamed-chunk-16-1} 

}

\caption{Typology of similarities}\label{fig:unnamed-chunk-16}
\end{figure}

\chapter{Conclusion}\label{conclusion}

\bibliography{book.bib}


\end{document}
